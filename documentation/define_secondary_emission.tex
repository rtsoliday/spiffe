%
\newpage

\subsection{define\_secondary\_emission}

\begin{itemize}

\item {\bf description:} Permits specification of secondary emission
yield function for the material surface.  This refers to emission of
one or more new particles when a particle impacts the surface.  By
default, no secondary emission is done.

\item {\bf example:} 
\begin{verbatim}
&define_secondary_emission
    input_file = ``secondary.sdds'',
    kinetic_energy_column = ``K'',
    yield_column = ``Yield''
&end
\end{verbatim}

\item {\bf synopsis and defaults:} 
\begin{verbatim}
&namelist secondary_emission
        STRING input_file = NULL;
        STRING kinetic_energy_column = NULL;
        STRING yield_column = NULL;
        long yield_limit = 0;
        double emitted_momentum = 0;
        long verbosity = 1;
        long material_id = 1;
        STRING log_file = NULL;
&end
\end{verbatim}

\item {\bf details:} 
\begin{itemize}
        \item {\tt input\_file} --- Name of an SDDS file from which the secondary emission
        yield curve will be read.
        \item {\tt kinetic\_energy\_column} --- Name of the column in {\tt input\_file} giving
        values of the particle kinetic energy, in eV.  
        \item {\tt yield\_column}  --- Name of the column in {\tt input\_file} giving
        values of the mean yield.  This is a ratio, giving the mean number of new electrons per
        incident electron.
        \item {\tt yield\_limit} --- If non-zero, this parameter limits from above the number
        of secondaries that can be emitted per primary particle. It can be helpful in preventing
        runaway, wherein the number of low-energy secondaries grows exponentially.
        \item {\tt emitted\_momentum} --- $\beta\gamma$ value for newly-emitted particles.
        The orientation of the momentum is random.
        \item {\tt verbosity} --- Larger positive values result in more detailed printouts
        during the run.
        \item {\tt material\_id} --- A positive integer giving the material for which this
          command specifies secondary emission properties. This will be used together with
          the \verb|material_id| parameter of the \verb|point| namelists in the geometry
          file to determine the appropriate secondary emission properties for each segment
          of the cavity boundary.
        \item {\tt log\_file} --- A possibly incomplete name of an SDDS file to which 
         secondary emission records will be written.
\end{itemize}

\end{itemize}

The algorithm is a simple one suggested by J. Lewellen (APS).  We
assume that the secondary emission yield is a function only of the
incident particle's kinetic energy.  Each time a particle is lost, the
code determines where the particle intersected the metal boundary.
The mean secondary yield is computed from the kinetic energy at the
time of loss.  The number of secondary particles emitted is chosen
using a Poisson distribution with that mean.  The secondary particles
are placed ``slightly'' ($\Delta r/10^6$ or $\Delta z/10^6$) outside
the metal surface.

To prevent runaway, the secondary yield curve should fall to zero for
low energies.  If you have problems with runaway, try setting the {\tt
yield\_limit} parameter to a small positive integer.  Runaway
appears to be associated (at times) with the occasional production of
large numbers of secondaries due to the tails of the Poisson
distribution.


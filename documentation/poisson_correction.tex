%
\newpage

\subsection{poission\_correction}

\begin{itemize}

\item {\bf description:}
This command requests correction of the electric fields so that they
satisfy Poisson's equation.  It is recommended in order to increase
the accuracy of the field solutions.  Without it, numerical errors
tend to build up that correspond to fictitious concentrations of
static charge on the grid.

\item {\bf example:} 
\begin{verbatim}
&poisson_correction
    start_time = 1e-9, 
    step_interval = 32,
    accuracy = 1e-4,
    error_charge_threshold = 1e-15
&end
\end{verbatim}
This requests that Poisson correction be performed every 32 simulation
steps starting 1ns after the start of the simulation.  The fractional
accuracy of the Poisson solver is $10^{-4}$.  It is invoked only when
the amount of fictitious charge exceeds 0.001 pC.

\item {\bf synopsis and defaults:} 
\begin{verbatim}
&poisson_correction
    double start_time = 0;
    long step_interval = 0;
    double accuracy = 1e-6;
    double error_charge_threshold = 0;
    long maximum_iterations = 1000;
    long verbosity = 0;
    double test_charge = 0;
    double z_test_charge = 0;
    double r_test_charge = 0;
    STRING guess_type = "none";
&end
\end{verbatim}

\item {\bf details:} 
\begin{itemize}
\item {\tt start\_time}: Time in seconds at which to begin Poisson correction.
\item {\tt step\_interval}: Interval between corrections in units of the simulation step.
\item {\tt accuracy}: Fractional accuracy of the Poisson solutions.
\item {\tt error\_charge\_threshold}: Amount of net error charge that must be present for
        Poisson correction to actually take place.
\item {\tt maximum\_iterations}: Maximum number of iterations of the relaxation loop for
        the Poisson solver.
\item {\tt verbosity}: Flag requesting informational output about Poisson correction.
\item {\tt test\_charge, z\_test\_charge, r\_test\_charge}:  For developmental use only.
\item {\tt guess\_type}: The model to use for the initial guess to start the Poisson
        iteration.  Possibilities are, ``none'', ``line-charge'', ``point-charge'', and
        ``zero.''  It is recommended that ``none'' be used.
\end{itemize}

\end{itemize}

%
\newpage

\subsection{translate}

\begin{itemize}

\item {\bf description:}
Sets up repeated translation of the memory used for the mesh in the +z
direction to follow the beam.  This can be used to simulate a long
drift while using less computing time and less memory.  There has been
little testing of this feature!

\item {\bf example:} 
\begin{verbatim}
&translate
        z_trigger = 0.05,
        z_lower = 0.02,
        z_upper = 0.055
&end
\end{verbatim}
This specifies that when the first particle passes $z=5$cm, the
mesh will be contracted to cover the region from $z=2$cm to 
$z=5.5$cm.  Thereafter, each time a particle advances by one
longitudinal mesh spacing, the grid will be translated that same
distance in the +z direction.

The boundary conditions are changed to Neumann on the upper, right, and
left boundaries and Dirichlet on the lower boundary.  

\item {\bf synopsis and defaults:} 
\begin{verbatim}
&translate
        z_lower = <zMin>,
        z_upper = <zMax>,
        z_trigger = <z0>,
&end
\end{verbatim}
where \verb|<zMin>| and \verb|<zMax>| are the minimum and maximum longitudinal
coordinates for the original problem mesh, and \verb|<z0>| is \verb|<zMax>|-$5*\Delta z$,
where $\Delta z$ is the mesh spacing.

\item {\bf details:} 
\begin{itemize}
    \item {\tt z\_trigger}:  The first time any particle passes the longitudinal
        position given for this variable, mesh reduction is triggered and
        translation is enabled.  Prior to this time, the {\tt translate} command
        has no effect.  After this time, translation by $\Delta z$ occurs 
        every time the maximum longitudinal particle position increases by
        $\Delta z$.
    \item {\tt z\_lower}: The lower limit of the longitudinal extent of the
        reduced mesh.
    \item {\tt z\_upper}: The upper limit of the longitudinal extent of the
        reduced mesh.
\end{itemize}

\end{itemize}

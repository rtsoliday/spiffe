%
\newpage

\subsection{define\_solenoid}

\begin{itemize}

\item {\bf description:}
Define solenoid coils, from which static magnetic fields will be
computed and imposed on the particle motion.  The {\tt
define\_solenoid} command permits defining a solenoid with a specified
length, radius, symmetry, and current/field.

\item {\bf example:} 
\begin{verbatim}
&define_solenoid
   radius = 0.05, 
   z_start = 0.01, z_end = 0.02,
   current = 1,
   Bz_peak = 1,
   turns = 100
&end
\end{verbatim}
This command defines a 100-turn solenoid extending longitudinally
between coordinates 1cm and 2cm, at a radius of 5cm.  The current,
initial 1A, is adjusted to obtain a peak on-axis longitudinal B field
of 1Tesla.

\item {\bf synopsis and defaults:} 
\begin{verbatim}
#namelist define_solenoid
        double radius = 0;
        double evaluation_radius_limit = 0;
        double z_start = 0;
        double z_end = 0;
        double current = 0;
        double Bz_peak = 0;
        long turns = 1;
        long symmetry = 0;
        STRING field_output = NULL;
        long bucking = 0;
        double z_buck = 0;
#end
\end{verbatim}

\item {\bf details:} 
\begin{itemize}
\item {\tt radius}: The radius of the coils in meters.
\item {\tt evaluation\_radius\_limit}: The maximum radius in meters at which the solenoidal
        fields should be computed.  This can save considerable CPU time if you
        are not interested in particles that go beyond a certain radius (e.g.,
        you know they'll be lost).
\item {\tt z\_start, z\_end}: The starting and ending longitudinal coordinate of the
        coils, in meters.
\item {\tt current}: The current in each coil in Amperes ({\bf not} Ampere-turns).
\item {\tt Bz\_peak}: The peak on-axis longitudinal magnetic field in Tesla desired
        from this solenoid.  The current is scaled to achieve this value.  Even if
        you give this value, you must give an initial value for {\tt current}.
\item {\tt turns}: The number of turns (or coils) in the solenoid.
\item {\tt symmetry}: Either 0, 1, or -1 for no symmetry, even symmetry, or odd
        symmetry.  For codes of $\pm 1$, the resulting fields are those produced
        by the combination of the specified solenoid and another identical solenoid
        extending from {\tt -z\_start} to {\tt -z\_end}.  If {\tt symmetry} is 1 (-1),
        the field from the mirror solenoid adds to (subtracts from) the field due
        to the specified solenoid.  
\item {\tt field\_output}: Requests output of the solenoid field to an SDDS file.
        This output is accumulated field from all solenoids defined up to this
        point, including the solenoid presently being defined.
\item {\tt bucking}: If nonzero, indicates that this is a bucking solenoid.  The solenoid current
        is adjusted to zero the on-axis value of $B_z$ at {\tt z\_buck}.
\item {\tt z\_buck}: If {\tt bucking} is nonzero, the location at which the field is bucked.
\end{itemize}

\end{itemize}

Here is an example of using three {\tt define\_solenoid} commands to make a solenoid field with
a peak on-axis $B_z$ value of 0.3T that is bucked to zero at $z=0$.  The main solenoid field is produced by
two sets of windings with a ratio of 10:1 between the current:
\begin{verbatim}

&define_solenoid
        radius = 0.06,
        evaluation_radius_limit = 0.1,
        z_start = 0.0,
        z_end = 0.15,
        current = 1,
        turns = 100,
&end

&define_solenoid
        radius = 0.06,
        evaluation_radius_limit = 0.1,
        z_start = 0.20,
        z_end = 0.30,
        current = 0.1,
        turns = 67,
&end

&define_solenoid
        radius = 0.02,
        evaluation_radius_limit = 0.1,
        z_start = -0.04,
        z_end = -0.02,
        current = 1,  ! any nonzero value will do
        turns = 100,
        bucking = 1,
        Bz_peak = 0.3,
        field_output = solenoid.sdds
&end
\end{verbatim}

%
\newpage

\subsection{integrate}

\begin{itemize}

\item {\bf description:}
Defines integration parameters and begins integration.  Allows
specifying the integration time step, the total time to integrate, and
other conditions of integration.

\item {\bf example:} 
\begin{verbatim}
&integrate
    dt_integration = 1e-12,
    start_time = 0,
    finish_time = 5e-9,
    status_interval = 128,
    space_charge = 1
&end
\end{verbatim}
Starts integration of equations for particles and fields
at a simulated time of 0, taking steps of 1ps, until reaching
5ns.  Every 128 steps, status information is printed to the
screen.  Space charge is included.

\item {\bf synopsis and defaults:} 
\begin{verbatim}
&integrate
    double dt_integration = 0;
    double start_time = 0;
    double finish_time = 0;
    long status_interval = -1;
    long space_charge = 0;
    long check_divergence = 0;
    double smoothing_parameter = 0;
    double J_filter_multiplier = 0;  
    long terminate_on_total_loss = 0;
    STRING status_output = NULL;
    STRING lost_particles = NULL;
&end
\end{verbatim}

\item {\bf details:} 
\begin{itemize}
    \item {\tt dt\_integration}: Simulation step size in seconds.
    \item {\tt start\_time}: Simulation start time.  Typically 0 for new runs.
        Ignored for runs that involve fields loaded from other simulations.
    \item {\tt finish\_time}: Simulation stop time.
    \item {\tt status\_interval}: Interval in units of a simulation step between 
        status printouts.
    \item {\tt space\_charge}: Flag requesting inclusion of space-charge in the
        simulation.
    \item {\tt check\_divergence}: Flag requesting that status printouts include
        a check of the field values using the divergence equation.
    \item {\tt smoothing\_parameter}: Specifies a simple spatial filter for the
        current density.  The smoothing parameter, $s$, is used to compute two
        new quantities, $c_1 = 1-s$ and $c_2 = s/2$.  The program smooths longitudinal
        variation for constant radius, using $A_o \rightarrow (A_- + A_+)c_2 + A_o c_1$,
        where $A_o$ is the central value and $A_\pm$ are the adjacent values to a grid point.
        This function is rarely used and I do not recommend it.
    \item {\tt J\_filter\_multiplier}: Specifies a simple time-domain filter for the
        current density.  For each point on the grid, the new current density value $J(1)$ is
        replaced by $J(1)*(1-f) + J(0)*f.  This is an infinite impulse response filter.
        This function is rarely used and I do not recommend it.
    \item {\tt terminate\_on\_total\_loss}: Flag requesting that when all simulation particles
        are lost (e.g., by hitting a wall or exiting the simulation), the simulation should
        terminate.
    \item {\tt status\_output}: Provide the name of a file to which to write status information,
        including statistics on the beams and fields.  File is in SDDS format.
    \item {\tt lost\_particles}: Provide the name of a file to which to write information about
        particles that get lost.  File is in SDDS format.
    \item {\tt auto\_max\_dt}: if non-zero, dt\_integration exceeding the maximum stable timestep is replaced by the maximum.
\end{itemize}

\end{itemize}

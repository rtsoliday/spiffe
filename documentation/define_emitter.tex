%
\newpage

\subsection{define\_emitter}

\begin{itemize}

\item {\bf description:}
Permits specification of properties of an emitting region on the surface.
Similar to \verb|define_cathode|, but without some features.
Unlike \verb|define_cathode|, the emitter can be any surface, not just a plane of
constant z.

\item {\bf example:} 
\begin{verbatim}
&define_emitter,
    material_id = 10
    current_density = 20e4, 
    start_time = 0, stop_time = 200e-12,
    electrons_per_macroparticle = 1e6
&end
\end{verbatim}

\item {\bf synopsis and defaults:} 
\begin{verbatim}
&define_emitter
    long material_id = -1;
    double current_density;
    double temperature;
    long determine_temperature;
    double work_function ; /* in eV */
    long add_thermal_velocities = 0;
    long field_emission = 0;
    double field_emission_beta = 1;
    double electrons_per_macroparticle;
    double start_time;
    double stop_time;
    double initial_p;
    long random_number_seed;
    int32_t halton_radix_dt = 0;
    int32_t halton_radix_r = 0;
    int32_t halton_radix_z = 0;
    STRING profile = NULL;
    STRING profile_time_name = NULL;
    STRING profile_factor_name = NULL;
    STRING emission_log = NULL;
#end

\end{verbatim}

\item {\bf details:} 
\begin{itemize}
        \item {\tt material\_id} --- A positive integer giving the material for which this
          command specifies emission properties. This will be used together with
          the \verb|material_id| parameter of the \verb|point| namelists in the geometry
          file to determine the appropriate emission properties for each segment
          of the cavity boundary.
    \item {\tt current\_density}: Base current density, in Amperes/${\rm m^2}$.
    \item {\tt temperature}: Temperature of the cathode in degrees Kelvin.  If
        zero, then emission is constant at the rate given by {\tt current\_density}.
        Otherwise, used together with the work function (given by the
        {\tt work\_function} parameter) and the Richardson-Schottky emission model
        to determine emission at each time step based on the electric field.
    \item {\tt work\_function}: Work function of the cathode material in eV.  Must
        be nonzero if the temperature is nonzero.
    \item {\tt determine\_temperature}: If nonzero, then attempts to determine the      
        temperature required to give the current density given by {\tt current\_density}.
        You must give the {\tt work\_function}.  Results are approximate because
        of the Richardson-Schottky law.
      \item {\tt add\_thermal\_velocities} : If nonzero, thermal velocities are
        added at the time of emission, assuming a Maxwellian velocity distribution.
    \item {\tt field\_emission}: If nonzero, then the cathode emits only by field
        emission.  The treatment of field emission is from section 6.13 of {\em
        The Handbook of Accelerator Physics and Engineering.}.  In field emission
        mode, {\tt spiffe} splits the cathode into many subcathodes, each one
        radial grid space in extent.  The field emission current density is
        computed for each subcathode separately, so that the results are correct
        in the case where the field varies over the cathode.
    \item {\tt field\_emission\_beta}: Gives the field enhancement factor for
        computing field emission current density.  The value of the electric
        field is multiplied by this factor before being used to compute the
        field emission current density.  Typical values are between 1 and 100.
        In this mode, you must specify {\tt electrons\_per\_macroparticle}.
    \item {\tt electrons\_per\_macroparticle}: Number of electrons represented by
        each macroparticle.  
    \item {\tt start\_time, stop\_time}: Start and stop time for emission, in seconds.
    \item {\tt initial\_p}:  The initial momentum of emitted
        particles, in dimensionless units (i.e., normalized to $m_e c$).
    \item {\tt random\_number\_seed}:  The seed for the particle emission random number
        generator.  A large, odd integer is recommended.  If 0 is given, the seed is picked based
        on the computer clock.
    \item {\tt spread\_over\_dt}: Flag requesting that emitted particles have their effective 
        emission times spread out over the simulation time step, $\Delta t$.  The particle velocities
        are adjusted appropriately using the instantaneous $E_z$ and $E_r$ fields {\em only}.
    \item {\tt halton\_radix\_t}: Halton radix (a small prime number) to be used for quiet-start
      generation of time values.
    \item {\tt halton\_radix\_r}: Halton radix (a small prime number) to be used for quiet-start
      generation of radius values.
    \item {\tt profile}: The name of an SDDS-protocol file containing a time-profile with which
        to modulate the base current density. 
    \item {\tt profile\_factor\_name, profile\_time\_name}: The columns giving the current-density
        adjustment factor and the corresponding time when it is valid for the file named by
        {\tt profile}.  The adjustment factor should be on $[0, 1]$.
    \item {\tt emission\_log}: The name of an SDDS-protocol file to which data will be written for
      each emitted particle.
\end{itemize}

\end{itemize}

%
\newpage

\subsection{define\_cathode}

\begin{itemize}

\item {\bf description:}
Permits specification of the size, current density, time profile, and
other parameters of particle emission from an annulus.  Starting in 
version 2.3, multiple cathodes may be defined.

\item {\bf example:} 
\begin{verbatim}
&define_cathode
    z_position = 0, outer_radius = 0.003,
    current_density = 20e4, 
    start_time = 0, stop_time = 200e-12,
    number_per_step = 8, 
&end
\end{verbatim}

\item {\bf synopsis and defaults:} 
\begin{verbatim}
&define_cathode
    double z_position = 0;
    double inner_radius = 0;
    double outer_radius = 0;
    double current_density = 0;
    double temperature = 0;
    double work_function = 0;
    long field_emission = 0;
    long add_thermal_velocities = 0;
    double field_emission_beta = 1;
    long determine_temperature = 0;
    double electrons_per_macroparticle = 0;
    double start_time = 0;
    double stop_time = 0;
    long autophase = 0;
    double time_offset = 0;
    double number_per_step = 0;
    double initial_pz = 0;
    double initial_omega = 0;
    double stiffness = 1;
    long discretize_radii = 0;
    long random_number_seed = 987654321;
    long distribution_correction_interval = 0;
    long spread_over_dt = 0;
    long zoned_emission = 1;
    int32_t halton_radix_dt = 0;
    int32_t halton_radix_r = 0;
    STRING profile = NULL;
    STRING profile_time_name = NULL;
    STRING profile_factor_name = NULL;
    STRING emission_log = NULL;
&end
\end{verbatim}

\item {\bf details:} 
\begin{itemize}
    \item {\tt z\_position}:  Longitudinal position of the cathode in meters.
    \item {\tt inner\_radius, outer\_radius}: Inner and outer radius of the edges
        of the cathode, in meters.
    \item {\tt current\_density}: Base current density, in Amperes/${\rm m^2}$.
    \item {\tt temperature}: Temperature of the cathode in degrees Kelvin.  If
        zero, then emission is constant at the rate given by {\tt current\_density}.
        Otherwise, used together with the work function (given by the
        {\tt work\_function} parameter) and the Richardson-Schottky emission model
        to determine emission at each time step based on the electric field.
    \item {\tt work\_function}: Work function of the cathode material in eV.  Must
        be nonzero if the temperature is nonzero.
    \item {\tt determine\_temperature}: If nonzero, then attempts to determine the      
        temperature required to give the current density given by {\tt current\_density}.
        You must give the {\tt work\_function}.  Results are approximate because
        of the Richardson-Schottky law.
      \item {\tt add\_thermal\_velocities} : If nonzero, thermal velocities are
        added at the time of emission, assuming a Maxwellian velocity distribution.
    \item {\tt field\_emission}: If nonzero, then the cathode emits only by field
        emission.  The treatment of field emission is from section 6.13 of {\em
        The Handbook of Accelerator Physics and Engineering.}.  In field emission
        mode, {\tt spiffe} splits the cathode into many subcathodes, each one
        radial grid space in extent.  The field emission current density is
        computed for each subcathode separately, so that the results are correct
        in the case where the field varies over the cathode.
    \item {\tt field\_emission\_beta}: Gives the field enhancement factor for
        computing field emission current density.  The value of the electric
        field is multiplied by this factor before being used to compute the
        field emission current density.  Typical values are between 1 and 100.
        In this mode, you must specify {\tt electrons\_per\_macroparticle}.
    \item {\tt electrons\_per\_macroparticle}: Number of electrons represented by
        each macroparticle.  
    \item {\tt number\_per\_step}: How many macroparticles to emit per
        step.  Incompatible with specifying {\tt electrons\_per\_macroparticle}.
    \item {\tt start\_time, stop\_time}: Start and stop time for emission, in seconds.
    \item {\tt autophase}: Flag requesting that cathode emission start only when the
        field at the cathode has the proper phase to accelerate the beam.  The total
        time for emission is still determined by the difference between the start
        and stop time.
    \item {\tt time\_offset}: Only relevant when {\tt autophase = 1}.  Specifies a time
        offset relative to the emission start time determined by autophasing.
    \item {\tt initial\_pz}:  The initial longitudinal momentum of emitted
        particles, in dimensionless units (i.e., normalized to $m_e c$).
    \item {\tt initial\_omega}: The initial angular velocity of particles, in
        radians per second.
    \item {\tt stiffness}: The beam stiffness, i.e., the particle mass,
        in electron masses.
    \item {\tt discretize\_radii}: Flag requesting that particles be emitted only from
        radii $n*\Delta r$, where n is an integer and $\Delta r$ is the radial grid spacing.
        This can be useful for certain types of diagnostic runs, but should not be used with    
        space charge.
    \item {\tt random\_number\_seed}:  The seed for the particle emission random number
        generator.  A large, odd integer is recommended.  If 0 is given, the seed is picked based
        on the computer clock.
    \item {\tt distribution\_correction\_interval}: The number of steps between corrections to
        the emitted particle distribution.  Can be used to compensate for nonuniform emission
        that occurs due to use of random numbers in the emission algorithm.  If used, it should
        be set to 1.  Cannot be used when the temperature is nonzero (Richardson-Schottky
        emission law).
    \item {\tt spread\_over\_dt}: Flag requesting that emitted particles have their effective 
        emission times spread out over the simulation time step, $\Delta t$.  The particle velocities
        are adjusted appropriately using the instantaneous $E_z$ and $E_r$ fields {\em only}.
    \item {\tt zoned\_emission}: Flag requesting that emission calculations take place separately
        for each annular zone of width $\Delta r$ (the radial grid spacing).  Reduces the possibility
        that random number effects will result in a nonuniform current density.
    \item {\tt halton\_radix\_dt}: Halton radix (a small prime number) to be used for quiet-start
      generation of time values.
    \item {\tt halton\_radix\_r}: Halton radix (a small prime number) to be used for quiet-start
      generation of radius values.
    \item {\tt profile}: The name of an SDDS-protocol file containing a time-profile with which
        to modulate the base current density. 
    \item {\tt profile\_factor\_name, profile\_time\_name}: The columns giving the current-density
        adjustment factor and the corresponding time when it is valid for the file named by
        {\tt profile}.  The adjustment factor should be on $[0, 1]$.
    \item {\tt emission\_log}: The name of an SDDS-protocol file to which data will be written for
      each emitted particle.
\end{itemize}

\end{itemize}

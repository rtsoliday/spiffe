%
\newpage

\subsection{define\_resistor}

\begin{itemize}

\item {\bf description:}
Defines a resistive element in the cavity interior.  These simulate
the effect of walls that are less than perfectly conducting.  It slows
the simulation tremendously, due to the reduced step size needed to
obtain numerical stability.

\item {\bf example:} 
\begin{verbatim}
&define_resistor
    start = 0.02, end = 0.03,
    position = 0.04, 
    direction = "z",
    conductivity = 6e3
&end
\end{verbatim}
This specifies a resistor extending in the longitudinal (z) direction
from 2cm to 3cm, at a radius of 4cm.  The conductivity is $6*10^3 {\rm
(m*ohms)}^{-1}$, the value for copper.


\item {\bf synopsis and defaults:} 
\begin{verbatim}
&define_resistor
    double start = 0;
    double end = 0;
    double position = 0;
    STRING direction = "z";
    double conductivity = 1e154;
&end
\end{verbatim}

\item {\bf details:} 
\begin{itemize}
    \item {\tt direction}:  The direction in which the resistor extends.
        May be ``z'' or ``r''.
    \item {\tt start, end}: The starting and ending coordinates of the
        resistor in the {\tt direction} dimension.
    \item {\tt position}: The position of the resistor in the ``other''
        dimension.  E.g., the radial position if {\tt direction} is ``z''.
    \item {\tt conductivity}: The conductivity of the metal in ${\rm
(m*ohms)}^{-1}$.

\end{itemize}

\end{itemize}

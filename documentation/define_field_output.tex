%
\newpage

\subsection{define\_field\_output}

\begin{itemize}

\item {\bf description:}
Requests output of the field map into an SDDS file.  A separate SDDS
data page is made for successive maps.  The interleaved simulation
field grids are interpolated to give all field values on the same
grid.

\item {\bf example:} 
\begin{verbatim}
&define_field_output
    filename = "gunRun.fields",
    time_interval = 0.1e-9,
    start_time = 1e-9,
    z_interval = 2,
    r_interval = 2
&end
\end{verbatim}
This command results in output of the fields on a grid that is twice
as coarse as the simulation grid in both z and r, starting at time
1ns and continuing at 100ps intervals thereafter.  The data is put
in an SDDS file named {\tt gunRun.fields}.

\item {\bf synopsis and defaults:} 
\begin{verbatim}
&define_field_output
    STRING filename = NULL;
    double time_interval = 0;
    double start_time = 0;
    long z_interval = 1;
    long r_interval = 1;
    long exclude_imposed_field = 0;
    long separate_imposed_field = 0;
&end
\end{verbatim}

\item {\bf details:} 
\begin{itemize}
    \item {\tt filename}: Name of an SDDS file in which to put the data.
    \item {\tt time\_interval}: Simulated time interval in seconds between successive maps.
    \item {\tt start\_time}: Starting simulated time for data output.
    \item {\tt z\_interval}: Interval at which grid points are spaced longitudinally in
        units of the simulation longitudinal grid size.
    \item {\tt r\_interval}: Interval at which grid points are spaced radially in
        units of the simulation radial grid size.
    \item {\tt exclude\_imposed\_field}: Flag requesting that any fields specified with the
        {\tt constant\_fields} command should be excluded from the field output.
    \item {\tt separate\_imposed\_field}: Flag requesting that any fields specified with the
        {\tt constant\_fields} command should be included in the output as separate data elements.
\end{itemize}

\end{itemize}

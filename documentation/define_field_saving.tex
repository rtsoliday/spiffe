%
\newpage

\subsection{define\_field\_saving}

\begin{itemize}

\item {\bf description:}
Field saving refers to writing the simulation fields to disk in 
such a way that they can be reloaded in subsequent runs.  At this
time, it saves only the time-dependent field components, not the
external fields defined with {\tt define\_solenoid} or {\tt constant\_fields}.

\item {\bf example:} 
\begin{verbatim}
&define_field_saving
    filename = "gunRun.fsave",
    time_interval = 1e-9,
    double start_time = 5e-9
&end
\end{verbatim}
This command results in saving of the fields at 1ns intervals starting
at simulated time of 5ns to file "gunRun.fsave".  Successive saves are
placed in successive SDDS pages.

\item {\bf synopsis and defaults:} 
\begin{verbatim}
&define_field_saving
    STRING filename = NULL;
    double time_interval = 0;
    double start_time = 0;
    long save_before_exiting = 0;
&end
\end{verbatim}

\item {\bf details:} 
\begin{itemize}
    \item {\tt filename}:  Name of an SDDS file to which to write the field saves.
    \item {\tt time\_interval}: Simulated time interval in seconds between successive saves.
    \item {\tt start\_time}: Simulated time at which to make the first save.
    \item {\tt save\_before\_exiting}: Flag requesting that a save be made just prior to
        program termination.
\end{itemize}

\end{itemize}

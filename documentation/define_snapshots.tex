%
\newpage

\subsection{define\_snapshots}

\begin{itemize}

\item {\bf description:}
Requests beam snapshots at equispaced time intervals.
These snapshots contain the spatial and momentum coordinates
of all particles.

\item {\bf example:} 
\begin{verbatim}
&define_snapshots
    filename = "gunRun.snap",
    time_interval = 0.1e-9
    start_time = 2e-9
&end
\end{verbatim}
This command results in snapshots being taken every 100ps starting
at time 2ns, with data going to {\tt gunRun.snap}.

\item {\bf synopsis and defaults:} 
\begin{verbatim}
&define_snapshots
    STRING filename = NULL;
    double time_interval = 0;
    double start_time = 0;
&end
\end{verbatim}

\item {\bf details:} 
\begin{itemize}
    \item {\tt filename}:  The name of an SDDS file to which to write the snapshots.
    \item {\tt time\_interval}:  The interval in seconds between successive snapshots.
    \item {\tt start\_time}: The time in seconds at which to make the first snapshot.
\end{itemize}

\end{itemize}

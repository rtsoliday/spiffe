%
\newpage

\subsection{define\_field\_sampling}

\begin{itemize}

\item {\bf description:}
Requests sampling of the fields vs time or space coordinates.  One may
set up sampling of a specific field component along a line in z or r,
with output vs the distance along the line at specified times or else
output of the average value along the line vs time.

\item {\bf example:} 
\begin{verbatim}
&define_field_sampling
    filename = "gunRun.Ez_t,
    component = "Ez",
    time_sequence = 1
    direction = "z",
    min_coord = 0, max_coord = 0.01,
    position = 0,
    time_interval = 10e-12,
    start_time = 0,
&end
\end{verbatim}
Requests logging of $E_z$ as a function of time, with samples made a
10ps simulated time intervals starting with the start of the
simulation.  The data output is the average of $E_z$ along a
line from $z=0$ to $z=1$cm.

\item {\bf synopsis and defaults:} 
\begin{verbatim}
&define_field_sampling
    STRING filename = NULL;
    STRING component = NULL;
    STRING direction = NULL;
    double min_coord = 0;
    double max_coord = 0;
    double position = 0;
    double time_interval = 0;
    double start_time = 0;
    long time_sequence = 0;
&end
\end{verbatim}

\item {\bf details:} 
\begin{itemize}
    \item {\tt filename}: Name of an SDDS file to which to write the data.
    \item {\tt component}: Field component to sample.  Must be one of
        "Ez", "Er", "Jz", "Jr", "Bphi", "Phi", "Ephi", "Bz", "Br", "Q",
        "Eall", or "Ball",
        E is the electric field, B is the magnetic field,
        J is the current density, Phi is the scalar potential, and Q
        charge assigned to a grid point.
        "Eall" and "Ball" request, respectively, all E- and B-field components.
        As of version 4.8.0, the E- and B-field outputs are obtained using
        the same interpolation routines as used for particle pushing, rather
        than merely taking the nearest grid points.
    \item {\tt direction}: The direction of the line along which samples are
        taken.  Must be "z" or "r".
    \item {\tt min\_coord, max\_coord}: The minimum and maximum coordinates
        of the sample line.
    \item {\tt position}: The position of the sample line in the direction
        orthogonal to {\tt direction}.
    \item {\tt time\_sequence}:  Flag requesting that instead of writing
        the selected component as a function of the coordinate {\tt direction},
        {\tt spiffe} instead write the average value of the component along
        the sample line as a function of time.  It zero, then {\tt spiffe}
        creates a new SDDS page for each sample time.
    \item {\tt start\_time}: The simulated time in seconds at which to start
        sampling.
    \item {\tt time\_interval}: The simulated time interval in seconds at which
        to make samples.
\end{itemize}

\end{itemize}

%
\newpage

\subsection{load\_fields}

\begin{itemize}

\item {\bf description:}

Allows loading fields from a previous {\tt spiffe} run.  The fields
are stored in a file created with the {\tt save\_fields} command.

\item {\bf example:} 
\begin{verbatim}
&load_fields
    filename = "fields.saved",
    factor = 1.5;
&end
\end{verbatim}
This loads fields from the file {\tt fields.saved}, applying a factor
of 1.5 to the values.  These fields become the only time-varying
fields in the problem.  (Others may be superimposed in subsequent
operations.)


\item {\bf synopsis and defaults:} 
\begin{verbatim}
#namelist load_fields
    STRING filename = NULL;
    double Ez_peak = 0;
    double factor = 1;
    double time_threshold = 0;
    long overlay = 0;
#end
\end{verbatim}

\item {\bf details:}
\begin{itemize}
\item {\tt filename}: Name of the file from which to take field
        data.  Normally created with the {\tt save\_fields} command.
        The file is in SDDS-protocol, typically with multiple data pages.
        Normally, the first page is used.  This may be modified with
        the {\tt time\_threshold} parameter.
\item {\tt Ez\_peak}: Desired maximum value of on-axis longitudinal electric
        field.  The fields from the data file are scaled to obtain this value.
        Note that this option cannot be used if TM fields are disabled.
        By default, no scaling occurs.
\item {\tt factor}: Factor by which to multiply the fields before use.
        If {\tt Ez\_peak} is 0, then this value is ignored.  One of
        {\tt factor} or {\tt Ez\_peak} must be nonzero.
\item {\tt time\_threshold}: Minimum simulation time, in seconds, at which
        the fields may have been created in order to be used.  For example,
        if {\tt time\_threshold} is {\tt 1e-9}, then {\tt spiffe} will advance
        through the pages of fields until it finds one from 1ps or more
        (in simulation time) after the start of the simulation that created
        the file {\tt filename}.
\item {\tt overlay}: Normally, the time-varying fields in the simulation are
        set equal to those in the file, within a scale factor.  If you wish to
        simply add the new fields to those already in force, then set {\tt overlay}
        to a nonzero value.
\end{itemize}

\end{itemize}

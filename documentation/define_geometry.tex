
\newpage

\subsection{define\_geometry}
\label{define_geometry}

\begin{itemize}

\item {\bf description:}
Define the simulation region and the geometry of a cavity.  This
includes defining the grid sizes, the boundary conditions, which types
of fields to include (TE or TM), and how to interpolate on the grid.
This is done with the {\tt define\_geometry} command, which requires
existence of a boundary definition file in a POISSON-like format.
This file specifies not only the location of the metal surfaces
in the problem, but optionally the potential of each.

\item {\bf example:} 
\begin{verbatim}
&define_geometry
    nz = 165, zmin = 0.0, zmax = 0.0959148653920, 
    nr = 165, rmax=0.04289777777,
    boundary = "mg6mm-3.geo", 
    boundary_output = "mg6mm-3.bnd",
    interior_points = "mg6mm-3.pts"
&end
\end{verbatim}
This command defines a problem with an extent of about 9.6cm
in the longitudinal direction and about 4.3cm in the radial
direction.  The number of grid lines in each dimension is
165.  Boundary input is taken from file {\tt mg6mm-3.geo}.
In addition, output of the boundary coordinates is to be
placed in file {\tt mg6mm-3.bnd} while output of the coordinates
of all interior grid points is to be placed in {\tt mg6mm-3.pts}.
 
\item {\bf synopsis and defaults:} 
\begin{verbatim}
&define_geometry
    long nz = 0;
    long nr = 0;
    double zmin = 0;
    double zmax = 0;
    double rmax = 0;
    double zr_factor = 1;
    STRING rootname = NULL;
    STRING boundary = NULL;
    STRING boundary_output = NULL;
    STRING urmel_boundary_output = NULL; 
    STRING discrete_boundary_output = NULL;
    STRING interior_points = NULL;
    STRING lower = "Dirichlet";
    STRING upper = "Neumann";
    STRING right = "Neumann";
    STRING left  = "Neumann";
    long include_TE_fields = 0;
    long exclude_TM_fields = 0;
    long turn_off_Er = 0;
    long turn_off_Ez = 0;
    long turn_off_Ephi = 0;
    long turn_off_Br = 0;
    long turn_off_Bz = 0;
    long turn_off_Bphi = 0;
    long print_grids = 0;
    long radial_interpolation = 1;
    long longitudinal_interpolation = 1;
    long radial_smearing = 0;
    long longitudinal_smearing = 0;
&end
\end{verbatim}

\item {\bf details:}
\begin{itemize}
\item {\tt nz, nr}: number of grid lines in z and r dimensions, respectively.
\item {\tt zmin, zmax}: starting and ending longitudinal coordinate, respectively.
\item {\tt rmax}: maximum radial coordinate.
\item {\tt zr\_factor}: a factor by which to multiply the z and r values in the boundary
 input file to convert the values to meters.  For example, {\tt zr\_factor=0.01} would be used if the 
 boundary input file values were in centimeters.
\item {\tt rootname}: rootname for construction of output filenames.  Defaults to the rootname of
 the input file.
\item {\tt boundary}: name of input file containing POISSON-like specification of the cavity boundary
        and surface potentials.
\item {\tt boundary\_output}: (incomplete) name of output file to which SDDS-protocol data will be sent containing
        the coordinates of points on the {\em ideal} boundary, i.e., the boundary you would get if
        your grid spacing was zero.  Recommended value: ``%s.bouti''.
\item {\tt discrete\_boundary\_output}: (incomplete) name of output
        file to which SDDS-protocol data will be sent containing the
        coordinates of points on the actual boundary used in the
        simulation.  This differs from the ideal boundary because
        every point on the actual boundary must be a grid point.
        Recommended value: ``%s.boutd''.
\item {\tt interior\_points}: (incomplete) name of output file to which SDDS-protocol data will be sent containing 
        the coordinates of all interior points of the cavity.  May be used together with boundary
        output files and {\tt sddsplot} to manually confirm the interpretation of the cavity
        specification by {\tt spiffe}.
\item {\tt lower, upper, right, left}: boundary conditions for the edges of the simulation region.
        The defaults are listed above.  ``Dirichlet'' boundary conditions means that electric field
        lines are parallel to the boundary.  ``Neumann'' boundary conditions means that electric field
        lines are perpendicular to the boundary.
\item {\tt include\_TE\_fields}: flag indicating whether to include transverse-electric fields, i.e.,
        fields having no longitudinal electric field components.  If you include space-charge and the
        beam is rotating, you should set this to 1.
\item {\tt exclude\_TM\_fields}: flag indicating whether to exclude transverse-magnetic fields, i.e.,
        fields having no longitudinal magnetic field components.  
\item {\tt turn\_off\_...}: flags indicating that the specified fields should be ``turned off,'' which
  means that the don't affect particles.  Used for testing purposes.
\item {\tt print\_grids}: flag requesting a text-based picture of the simulation grids.
\item {\tt radial\_interpolation, longitudinal\_interpolation}: flags requesting that field components be 
        interpolated in the radial and longitudinal direction when fields are applied to particles.  If 0,
        then fields will change abruptly as particles move from one grid square to the next.  
\item {\tt radial\_smearing, longitudinal\_smearing}: flags requesting that charge and current from
        simulation macro particles be smeared over the grid points surrounding each particle.
        If 0, charge and current are assigned to the nearest grid point.
\end{itemize}

\end{itemize}

%
\newpage

\subsection{define\_screen}

\begin{itemize}

\item {\bf description:}
Defines a series of diagnostic ``screens'' for placement in the beam
path.  A single {\tt define\_screen} command can be used to produce
output of beam parameters at a series of equispaced longitudinal
positions.

\item {\bf example:} 
\begin{verbatim}
&define_screen
    template = "gunRun-%03ld.sdds",
    z_position = 0.01,
    delta_z = 0.01,
    number_of_screens = 5,
    start_time = 1e-9,
    direction = "forward"
&end
\end{verbatim}
Defines a series of 5 screens for detection of forward-moving
particles, The screens are positioned at z coordinates of 1, 2, 3, 4,
and 5cm, with data going to files named {\tt gunRun-000.sdds}, {\tt
gunRun-001.sdds}, etc.  Detection begins only after 1ns of simulated
time.

\item {\bf synopsis and defaults:} 
\begin{verbatim}
&define_screen
    STRING filename = NULL;
    STRING template = NULL;
    double z_position = 0;
    double delta_z = 0;
    long number_of_screens = 1;
    double start_time = 0;
    STRING direction = "forward";
&end
\end{verbatim}

\item {\bf details:} 
\begin{itemize}
    \item {\tt filename}: Name of an SDDS file to which to write the data.  Used only
        if {\tt number\_of\_screens = 1}.
    \item {\tt template}: Template for creating the names of SDDS files to which to write
        data.  The template must contain a C-style format specifier for a long integer.
        The template is used as an argument to the C function {\tt sprintf} to create each
        filename in turn.
    \item {\tt z\_position}: The position (or starting position) of the screen (or series of
        screens), in meters.
    \item {\tt delta\_z}: In {\tt number\_of\_screens} is greater than 1, gives the distance in
        meters between successive screens.
    \item {\tt number\_of\_screens}: The number of screens in the series.
    \item {\tt start\_time}: The starting time in seconds for accumulation of data.
    \item {\tt direction}: The direction in which particles must be traveling in order to
        be recorded by the screen.  May be "forward" or "backward".
\end{itemize}

\end{itemize}

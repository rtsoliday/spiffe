%
\newpage

\subsection{Geometry File}
\label{geometryFile}

\begin{itemize}

\item {\bf description:}
This page describes the structure of the geometry file used with the
{\tt define\_geometry} namelist.  The file is similar to those used
with the program POISSON.

\item {\bf example:} 
\begin{verbatim}
! new RF gun, first cell
 &po x=0.000, y=0.000 &end
 &po x=0.000, y=0.006 &end
 &po x=0.0016, y=0.0135 &end
 &po x=0.0016, y=0.0427755  &end
 &po x=0.0060, y=0.0427755   &end
 &po nt=2, x0=0.006, y0=0.0181355, 
           r=0.02464, theta=0.0  &end
 &po x=0.03064  , y=0.014818    &end
 &po nt=2, x0=0.02861, y0=0.014818, 
           r=0.00203, theta=-90.0  &end
 &po x=0.02836, y=0.012788   &end
 &po nt=2, x0=0.02836, y0=0.010588, 
           r=0.0022, theta=180.0  &end
 &po x=0.02616, y=0.01008     &end
 &po nt=2, x0=0.03116, y0=0.01008, 
           r=0.005, theta=270.0  &end
 &po x=0.0348, y=0.00508  &end
 &po x=0.0348, y=0.00  &end
 &po x=0.000000, y=0.00  &end
\end{verbatim}

\item {\bf synopsis and defaults:} 
\begin{verbatim}
&point
    int nt = 1;
    double x = 0;
    double y = 0;
    double x0 = 0;
    double y0 = 0;
    double r = 0;
    double a = 0;
    double b = 0;
    double aovrb = 0;
    double theta = 0;
    int change_direction = 0;
    double potential = 0;
    int ramp_potential = 0;
    int material_id = 0;
&end
\end{verbatim}

\item {\bf details:}
\begin{itemize}
\item {\tt nt}: The segment type, where 1 (the default) indicates a line segment;
        2 indicates an arc of a circle; 3 indicates the start of a separate
        structure; and 4 indicates a definition of an in-vacuum point.
\item {\tt x, y}: For {\tt nt=1}, the endpoint of the line.  x corresponds to z 
        (the longitudinal coordinate) and y corresponds to r (the radial coordinate).
        For {\tt nt=2}, the endpoint of the circular or elliptical arc relative to (\verb|x0|, \verb|y0|).
        For {\tt nt=3}, the first point on the new shape.  For {\tt nt=4},
        the coordinates of the in-vacuum point.
\item {\tt x0, y0}: For {\tt nt=2}, the center of the circular or elliptical arc.
\item {\tt r}: For {\tt nt=2}, the radius of the circular arc. The equation of the arc is
$(x-x_0)^2 + (y-y_0)^2 = r^2$.
\item {\tt a}, {\tt b}, {\tt aovrb}: For {\tt nt=2}, the parameters of the elliptical arc.
The equation of the arc is 
$(x-x_0)^2/a^2 + (y-y_0)^2/b^2 = 1$.
If \verb|a| or \verb|b| is zero, then the value is determined using the ratio \verb|aovrb| (a over b).
\item {\tt theta}: For {\tt nt=2}, the angle in degrees of the end of the arc as seen from the
        center of the arc.  If this angle is less (greater) than the angle of the starting
        point (which is on $[-180, 180]$), then the sense of the arc is clockwise (counter-clockwise).
\item {\tt change\_direction}: For \verb|nt=2| when \verb|theta| is not given but \verb|x| and \verb|y| are,
  \verb|spiffe| may have trouble determining the direction of the arc. This flag can be used to change the
  direction.
\item {\tt potential}: The potential of the segment, in volts.
\item {\tt ramp\_potential}: If non-zero, the potential along a line is ramped linearly between the
  given value and the end value for the previous segment.  Only implemented for {\tt nt=1}.
\item \verb|material_id|: a positive integer that identifies the material of which this segment is
  made. Used to associate segments of the geometry with primary emission (\verb|define_emitter|) and
  secondardy emission (\verb|define_secondary_emission|).
\end{itemize}

\end{itemize}
